\section{Orientation du projet}
    Après 2 semestres d'ECAO Rubik's Cube, nous avons à présent un robot physique fonctionnel, ainsi qu'un programme procurant une solution à celui-ci de niveau débutant. Cette solution est calculée par la classe \texttt{EasyResolution} et est composée en moyenne de 150 mouvements humains (logique métier), ce qui se traduit par environ 500 mouvements robot et une résolution par celui-ci d'une durée moyenne de 30mn.
    
    L'objectif premier de ce semestre est d'améliorer le programme de résolution sur deux aspect : propreté et réutilisabilité du code ainsi qu'une amélioration des performances du robot en ce qui concerne le temps d'exécution de la solution.
    
    Le premier objectif s'est traduit par le choix de créer une base d'algorithmes externalisée par rapport au reste du code.
    
    En ce qui concerne le deuxième objectif, le problème est le suivant : le robot met actuellement 30 minutes en moyenne pour effectuer une résolution du Rubik's Cube. Une telle durée est acceptable pour un modèle de résolution développé en deux semestre ; cependant, pour les démonstrations au grand public, cette durée peut être longue, particulièrement lorsque le public est composé de non connaisseurs ou d'enfants. Ceci le deuxième objectif : l'amélioration des performances du robot en termes de temps total pour effectuer les mouvements composant l'algorithme solution.
    
    ~
    
    On notera que l'implémentation de la première solution facilitera l'atteinte du second objectif car il sera plus aisé d'ajouter de nouveaux algorithmes dans la base de donnés.
    
    
\section{Étapes temporelles du projet -- historique}
    Ci-dessous se trouve un résumé chronologique des différentes étapes de l'évolution du projet durant le semestre :
    
    ~
    
    \noindent
    \begin{tabular}{p{.25\textwidth} p{.65\textwidth}}
    \textbf{01/03 $\rightarrow$ 22/03} :
    & 
    $\bullet$ Approche bottom-up pour la création de la base de données :
    \begin{itemize}
     \item Création d'un modèle de base de donné format XML ;
     \item Création des DTD et XSD associées.
    \end{itemize}
    \\    
    \textbf{22/03 $\rightarrow$ 12/04} :
    &
    $\bullet$ Mise à jour de l'UML complet du projet ;
    
    $\bullet$ Début du nettoyage du code existant ;
    
    $\bullet$ Choix utilisation SAX ou DOM pour manipulation XML.
    
    ~
    
    \\
    \textbf{12/04 $\rightarrow$ 31/05} : 
    &
    $\bullet$ Création de la classe ayant pour rôle d'interfacer la BD(XML) avec l'application Java;
    
    $\bullet$ Création de la résolution \texttt{MediumResolution} (niveau intermédiaire).
    
    \\
    \textbf{31/05 $\rightarrow$ 15/06} : & $\bullet$ Rédaction du rapport d'activité incluant l'état actuel du projet complet.
    \end{tabular}

    
