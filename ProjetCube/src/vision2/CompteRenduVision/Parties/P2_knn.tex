  \subsubsection{Principe}

  Cette méthode est simple et nous a permis d'avoir un premier aperçu du rendu d'une classification. 

  Nous avons utilisé les données des centres pour créer notre modèle. 

  Ainsi les éléments vont être rattachés à la classe caractérisée par le centre le plus proche. 
Nous avons utilisé la distance euclidienne pour calculer la distance entre les éléments. 

  \subsubsection{Implantation initiale}

  Nous avons utilisé la toolbox créée en cours de Data Mining dont la signature est : 

  \verb|function [y_new] = kppv( x_new,x, y)|

  On utilise les valeurs de paramètres suivants : 
  \begin{itemize}
    \item x\_new = valeur RGB des carrés ;
    \item x = valeur RGB des centres ;
    \item y = label associé à chaque centre. 
  \end{itemize}

Les valeurs calculées sont : 
  \begin{itemize}
    \item y\_new = label des différents éléments de x. 
  \end{itemize}
