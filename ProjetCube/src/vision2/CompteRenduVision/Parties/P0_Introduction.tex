  Le but de cette UV est de compléter le travail de résolution du \rubic{} effectué par nos camarades au cours des semestres précédents en 
y ajoutant un module de reconnaissance de formes. 
  Ce module aura pour but de reconnaître la configuration de départ du \rubic{} afin de rendre totalement automatisée la chaîne d'acquisition. 

  Nous allons diviser cette tache en trois tâches successives : 
\begin{enumerate}
  \item Recalage permettant de trouver les carrés de la face considérée,  
  \item Détection des couleurs à partir des images de la webcam couleur, 
  \item Interfaçage avec le système de résolution déjà existant en Java. 
\end{enumerate}

  Nous avons choisi de traiter ce sujet car nous voulions approfondir et mettre en application les notions de traitement d'images 
et reconnaissance de forme. 
  De plus, l'automatisation du robot permettrait de mettre en valeur le travail des élèves du département 
\asi{} par la finition de la chaîne de traitement des données aboutissant à la résolution du \rubic{}. 

  Nous avons travaillé sur des images prises grâce à une webcam fixée sur le robot. 
Nous avons ainsi récupéré les différentes face du \rubic{} avec des conditions environnementales notamment d'éclairage différentes. 

  Nous avons choisi de travailler sur ces images sous Octave. 
En effet, l'outil Octave est étudié pour optimiser la manipulation des matrices et est libre, 
c'est pour cela que nous avons choisi cet outil de développement. 

  Nous allons dans la suite de ce rapport vous décrire notre démarche et justifier nos choix pour chacune des trois tâches 
décrites ci-dessus. 
