Afin de rendre notre code compréhensible, nous avons fait un inventaire des fonctions proposées 
par Octave que nous avons eu à utiliser : 

\begin{center} 
\begin{tabular}{|c|c|}
  \hline 
    diff & calcul la dérivée du vecteur en entrée \\
    &(la dérivée a une valeur de moins que le vecteur en entrée)\\
  \hline 
    sign & retourne un vecteur contenant\\
    & 1 si la valeur en entrée est positive\\
    & -1 si la valeur en entrée est negative\\
  \hline 
    find & retourne les indices tel que la condition en entrée soit vrai\\
    &(cette condition concerne souvent des vecteurs)\\
  \hline 
    min, max & \\
  \hline 
    abs & \\
  \hline 
\end{tabular}
\end{center}
