  La première étape est le recalage de l'image permettant de trouver les carrés de la face considérée. 

  Pour cela, nous avons étudié et comparé différentes façons de trouver les contours du \rubic{} : 
\begin{itemize}
  \item Faire une rotation de l'image pour recadrer l'image, 
  \item Étudier les projetés en X et Y des couleurs des points de l'image pour trouver les lignes de points 'noirs' caractéristiques des 
contours : 
    \begin{itemize}
      \item Par l'étude de la dérivée, 
      \item Par la recherche par seuil, 
      \item Par l'étude de la variance des N points successifs. 
    \end{itemize}
  \item Étudier la variation sur une fenêtre de points de l'image. 
\end{itemize}

  Nous avons travaillé sur une image en niveaux de gris car l'information sur les couleurs n'était pas intéressante pour cette étude. 
Nous avons obtenu l'image en niveaux de gris grâce aux fonctions \textbf{imread} et \textbf{rgb2gray} disponible sous octave. 
