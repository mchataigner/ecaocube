  Nous avons comparé ces différentes méthodes afin de choisir la plus appropriée à nos besoins. 

  Afin de tester la performance de chacune de ces méthodes, nous avons testé chacune d'elle sur un échantillon de photo de 5 lots de 6 faces avec des prétraitements différents.

\subsection*{Etude des pré-traitements adaptés pour chaque méthode}

  Dans ce paragraphe, nous allons présenté les choix des pré-traitements le mieux adaptés à chacune des méthode précédemment étudiées. 

%%%%%%%%%%%%%%%%%%%%%%%%%%%%%%%%%%%%%%%%%%%%%%%%%%%%%%%%%%%%%%%%%%%%%%%%%%%%%%%%%%%%%%%%%%%%%%%%%%%%%%%%%%%%%%%%%%%%%%%%%%%%%%%%%%%%%%%%%%%%%%%%%%%%%%%%%%%%%%%% DERIVEE
  Voilà les résultats obtenus pour cet échantillon par la méthode de la dérivée : \\
\begin{tabular}{|c|c|c|c|c|}
  \hline
 \textbf{Prétraitement} & NF + NO & F + NO & NF + O & F + O  \\
  \hline
  OK & 89\% & 94\% & 83\% & 94\% \\ % conf 4 5 6 
  OKR & 45\% & 55\% & 25\% & 25\% \\  % conf 7 8
  \hline 
\end{tabular}

F = utilisation d'un filtre, NF = pas d'utilisation d'un filtre, O = image orientée, NO = image non orienté \\
OK = nombre d'image où la détection des contours est un succés, OKR = nombre d'image ayant des reflets où la détection des contours est un succés\\

Nous comptons comme succes le fait de trouver les deux séparations centrale entre les facettes
  
  Nous constatons que cette méthode donne généralement de bon résultats. En effet la majore partie des échecs sont du à la détection d'une arrète du cube au lieux d'une séparation.  % A remplir 


%%%%%%%%%%%%%%%%%%%%%%%%%%%%%%%%%%%%%%%%%%%%%%%%%%%%%%%%%%%%%%%%%%%%%%%%%%%%%%%%%%%%%%%%%%%%%%%%%%%%%%%%%%%%%%%%%%%%%%%%%%%%%%%%%%%%%%%%%%%%%%%%%%%%%%%%%%%%%%%% SEUILLAGE
  Voilà les résultats obtenus pour cet échantillon par la méthode du seuillage : \\
\begin{tabular}{|c|c|c|c|c|}
  \hline
 \textbf{Prétraitement} & NF + NO & F + N0 & NF + 0 & F + 0  \\
  \hline
  OK & 100\% & 94\% & 89\% & 100\% \\ % conf 4 5 6 
  OKR & 83\% & 83\% & 83\% & 75\% \\ % conf 7 8
  \hline 
\end{tabular}

F = utilisation d'un filtre, NF = pas d'utilisation d'un filtre, O = image orientée, NO = image non orienté \\
OK = nombre d'image où la détection des contours est un succés, OKR = nombre d'image ayant des reflets où la détection des contours est un succés\\

Nous comptons comme succes le fait de trouver au moins les deux séparations centrales des facettes.

 Nous constatons que les résultats sont glogalement bon. Cependant il apparait un bruit sous la forme d'une zone sombre parazite qui peut être très genant pour la suite tu travail.


%%%%%%%%%%%%%%%%%%%%%%%%%%%%%%%%%%%%%%%%%%%%%%%%%%%%%%%%%%%%%%%%%%%%%%%%%%%%%%%%%%%%%%%%%%%%%%%%%%%%%%%%%%%%%%%%%%%%%%%%%%%%%%%%%%%%%%%%%%%%%%%%%%%%%%%%%%%%%%%% VARIANCE 1D 
  Voilà les résultats obtenus pour cet échantillon par la méthode de la variance sur les projections en X et Y : \\
\begin{tabular}{|c|c|c|c|c|}
  \hline
 \textbf{Prétraitement} & NF + NO & F + N0 & NF + 0 & F + 0  \\
  \hline
  OK & 0\% & 0\% & 0\% & 0\% \\ % conf 4 5 6 
  OKR & 0\% & 0\% &  0\% & 0\% \\ % conf 7 8
  \hline 
\end{tabular}

F = utilisation d'un filtre, NF = pas d'utilisation d'un filtre, O = image orientée, NO = image non orienté \\
OK = nombre d'image où la détection des contours est un succés, OKR = nombre d'image ayant des reflets où la détection des contours est un succés\\

Nous comptons comme succes le fait que chaques intersection tombe sur une facette colorée.

  En vue des résultats obtenus, nous pouvons clairement dire que cette méthode ne se suffit pas à elle même. Cependant, il est possible de combiner cette approche avec la méthode de la dérivé en vu d'amméliorer ces résultats.

%%%%%%%%%%%%%%%%%%%%%%%%%%%%%%%%%%%%%%%%%%%%%%%%%%%%%%%%%%%%%%%%%%%%%%%%%%%%%%%%%%%%%%%%%%%%%%%%%%%%%%%%%%%%%%%%%%%%%%%%%%%%%%%%%%%%%%%%%%%%%%%%%%%%%%%%%%%%%%%% VARIANCE 2D 
  Voilà les résultats obtenus pour cet échantillon par la méthode de la variance sur l'image : \\
\begin{tabular}{|c|c|c|}
  \hline
 \textbf{Prétraitement} & NO & 0  \\
  \hline
  OK & 55\% & 55\%  \\ % conf 4 5 6 
  OKR & 0\% & 0 \% \\ % conf 7 8
  \hline 
\end{tabular}

O = image orientée, NO = image non orienté \\
OK = nombre d'image où la détection des contours est un succés, OKR = nombre d'image ayant des reflets où la détection des contours est un succés\\

  Ces résultat montrent que cette méthode implémenté seule ne permet pas une detection correcte des facettes du cube.

  
